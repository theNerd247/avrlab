%Objectives of report: To create a basic thermometer using Atmel328p

%Methodology: Thermistor -> MCU -> 7 SEG
%Program Method: 2 Interrupts with linear interpolation

%Results: Working resistor tested range between 0-30 deg. cel. 

\documentclass[main.tex]{subfiles}
\begin{abstract}

A thermometer is a device used to measure and display the temperature of an
external environment. A typical thermometer can be created using a Atmega328p
microcontroller, thermistor and seven segment display. The main challenge in
this design is measuring and storing the temperature in the microcontroller. A
ten-bit analog-to-digital converter (ADC) in the Atmega328p can be used to
overcome this challenge. Temperature is converted to an analog voltage signal
by the thermistor which is then converted to a binary number by the ADC and is
stored in the Atmega328p. Linear interpolation is then performed on this number
to obtain a temperature value which is then displayed via a two digit seven
segment display. Testing of this method showed that the thermometer worked for
temperatures ranging from zero to forty degrees Celsius.
\end{abstract}

