\documentclass[a4paper,titlepage]{report}

%include the packages and styles to use
%\usepackage{package_name}
\usepackage{subfiles}
\usepackage{report}

\begin{document}

\title{Basic Analog to Digital Thermometer using ATmega328p}
\author{Douglas Smith, Noah Harvey}

% Date is automagically included (see Latex documentation)
\maketitle

\begin{abstract}
	A thermometer is a device used to measure and display the temperature of an
external environment. A typical thermometer can be created using an Atmega328p
microcontroller, thermistor and seven segment display. In the design presented
temperature is converted to an analog voltage signal by the thermistor. The
voltage signal is then converted to a binary number by an analog to digital
converter and is stored in the Atmega328p. Linear interpolation is then
performed on this number to obtain a temperature value which is then displayed
via a two digit seven segment display. The temperature conversion and display
tasks are controlled asynchronously via two interrupt service routines.  Testing
of this method showed that this design worked for temperatures ranging from 
$0-40^{\circ}\mathrm{C}$.

\end{abstract}

%\tableofcontents
\pagebreak

\subfile{introduction}
\subfile{hardware}
\subfile{program}
\subfile{results}
%\subfile{appendix}

\end{document}
