\documentclass[main.tex]{subfiles}
\begin{document}
\chapter{Introduction}
	\section{The Atmega328p}
	%TODO: write brief description of Atmega328p and the used features

	\section{Analog To Digital Conversion}
	%TODO: clean this section to fit into doc.
	\label{sec:ADC}
	The Atmega328p has a ten bit successive approximation analog to digital
	converter (noted as ADC). Successive approximation is an analog to digital
	conversion method in which the input signal is constantly compared to the
	output of a guessed analog value. (see \figref{adcSA} for a simplified flow
	diagram of a successive approximation ADC.). After all n bits of the converter
	are set the ADC interrupt service routine (or ADC ISR) is called to handle
	temperature conversion (see the ADC to Temperature Conversion section below). 

	\begin{figure}[H]
		\begin{center}
			\includegraphics[width=\linewidth]{adc}
		\end{center}
		\caption{ADC Successive Approximation}
		\label{fig:adcSA}
	\end{figure}
		
	
	\section{Temperature Measurement}
	A thermistor is a resistor in which its resistance is a function of
	temperature. When the thermistor is placed in a parallel circuit as shown in
	\figref{parallelThermistor} the value of $V_{out}$ become a function (see
	\eqref{Vout})of $R_{1}$ (the thermistor). Because $R_{1}$ varies with
	temperature this makes $V_{out}$ a function of temperature as well and can
	then be used to calculate a ``sensed'' temperature.
	
	\begin{equation}
		V_{out}(R_{1}) = \frac{V_{cc}R_{2}}{R_{1}+R_{2}}
		\label{eq:Vout}
	\end{equation}
	
	\begin{figure}[H]
		\begin{center}
			\includegraphics[width=0.25\linewidth]{thermParallel}
		\end{center}
		\caption{Thermistor ($R_{1}$) in Parallel}
		\label{fig:parallelThermistor}
	\end{figure}
	
	There does not exist a universal function that maps voltage to temperature.
	Instead a table is created and once a voltage is measured linear interpolation
	is used to calculate a temperature. \tabref{sampleData} shows a sample table.
	The data in the temperature and resistance columns are unique to the
	thermistor being used and can obtained from the thermistor's datasheet. The
	data in the voltage column was created using~\eqref{Vout} (where $V_{cc} = 5$
	and $R_{1} = 1\mathrm{k}\Omega$). The ADC column data are the resulting ADC values
	for their corresponding $V_{out}$.

	%TODO: add units to headingsheadings
	\begin{table}[H]
		\begin{center}
			\csvautobooktabular[
				after table=\caption{Sample Temperature Conversion Data}\
					\label{tab:sampleData}
			]{../sampleTemp.csv}
		\end{center}
	\end{table}
	
	\section{Linear Interpolation}
	%TODO: rewrite to flow with previous section
	\label{sec:linearInterpolation} 
	
	Linear interpolation is a method of estimating the values in between entries
	of a data table (in this case \tabref{sampleData}).  To find interpolated
	values based on a desired value (in this case an ADC value) the two
	``nearest'' data points are chosen and a linear function is formed between the
	two using point-slope form:

	\[
		f(x) = \frac{(y_{2}-y_{1})}{(x_{2}-x_{1})}(x-x_{1})+y_{1}\\
	\]

	(where $\left(x_{1},y_{1} \right)$ and $\left(x_{2},y_{2}\right)$ are two
	nearest data points). 

	%The idea is two points are chosen from the table (one
	%greater than, one less than the desired value) and a linear equation is formed
	%from those two points. Then the desired function value is found by computing
	%the formed function with a desired given input value into this function.
	
	%include graph of data
	\begin{figure}[H]
		\begin{center}
			\subfile{figures/adcTempGraph.tex}
		\end{center}
		\caption{Linear Interpolation of~\tabref{sampleData}}
		\label{fig:adcTempGraph}
	\end{figure}
	
	\figref{adcTempGraph} shows the linear interpolation of~\tabref{sampleData}.
	ADC values are used instead of $V_{out}$ because this is what the ATmega328p
	translates $V_{out}$ into (therefore converting ADC values into a voltage
	before performing a linear interpolation is inefficient).
	
	The lines between each data point represent
	the functions to be used for linear interpolation. Here is an example of
	computing the temperature for the ADC value $400$.
	
	To perform a linear interpolation for the ADC value $400$ one
	would find the two nearest ADC values to $400$ in the table ($(343,20)$ and
	$(445,30)$) and create the linear function between the two. Substituting $400$ for $x$
	gives the corresponding temperature value ($25.59^{\circ}\mathrm{C}$).
	
	\begin{eqnarray*}
		f(x) & = & \frac{(30-20)}{(445-343)}(x-343)+20 \\
		& \therefore & f(400) \approx 25.59^{\circ}\mathrm{C}
	\end{eqnarray*}
	
	A generic equation would be:
	

	Each line in \figref{adcTempGraph} shows the linear
	interpolation for the corresponding table points\footnote{Note: there are no
		interpolation lines past the end points of the table ($(241,10)$ and
		$(938,90)$). Interpolation beyond these points can be made by continuing
		their previous lines, however this could result in under/over
		approximation which can be dangerous in some  applications.}.
	
	
	%This report is divided into two sections; the first half is for program logic
	%and the second is for hardware. The software section will cover configuration of
	%the ATmega328p's timer for updating the seven-segment display and digit
	%multiplexing, the ADC for converting the thermistors voltage signal into digital
	%data and extracting the thermistor temperature from the data. The hardware
	%extion will cover setting up the Dragon programmer, powering the Atmega and the
	%ADC, seven-segment pin layout and setup, and connecting the thermistor.
	
\end{document}
