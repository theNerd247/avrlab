%        File: program.tex
%     Created: Thu Nov 06 08:00 AM 2014 E
% Last Change: Thu Nov 06 08:00 AM 2014 E
%
\documentclass[a4paper]{article}
\usepackage{listings}

\begin{document}
\chapter{Program Logic} 
	\section{Overview}
	The controlling program uses two interrupts to control the external display and
	temperature conversion tasks of the program asynchronously. An eight bit timer
	and ten bit analog to digital converter were used to control the calling of the
	above interrupts. See \ref{fig:progLogic}
	
	This project used the AVR version of the C programming language (ANSI C with
	binary number support). The AVR development suite (avr-dude, avr-gcc, etc) was
	used to compile and download the program to the microcontroller. 

	\begin{figure}[h]
		\begin{center}
			\psfig{figure=../figures/ProgramLogic.jpg}
		\end{center}
		\caption{Thermometer Program Logic}
		\label{fig:progLogic}
	\end{figure}


	\section{Microcontroller Features Used}
	The Atmega328p has a ten bit successive approximation analog to digital
	converter (noted as ADC). Successive approximation is a analog to digital
	conversion method in which the input signal is constantly compared to the output
	of the currently guessed analog value (which is stored digitally in an n-bit
	register and is then converted to an analog signal using a digital to analog
	converter). After all n bits of the converter are set the ADC raises an
	conversion completion interrupt flag. It is then that the ADC Interrupt Service
	Routine (noted as ISR) is called to handle temperature conversion (see the ADC
	to Temperature Conversion section below).
	
	An eight bit timer was used to handle the temperature display. The timer was
	configured to run at approximately fifty-two kilohertz in Clear Timer on Compare
	(noted as CTC) mode. CTC mode is an operation mode of the timer in which the
	timer's internal counter is reset when the counter matches the value stored in
	the timer's output compare register. Once the timer is reset a timer interrupt
	vector flag is set and the timer ISR is called to handle the display (see the
	Two Digit display section below).

	\section{Two Digit Display}
		\subsection{Two Digit Display Multiplexing}
		To display a two digit number a multiplexing method is used. PORTC on the
		Atmega328p is used to control which digit on the seven segment display is
		activated. When the digits are alternatively toggled at a frequency larger
		than 10kHz they appear as if they are displaying simultaneously. The timer  
		ISR handles the toggling of the displays of the two digits.


		\subsection{Single Digit Display Control}
		The two digit seven segment display is controlled by the Atmega328p via two GPIO
		ports. PORTD controls the digit to display and PORTC controls the digit
		selection. An array maps the digits 0 through 9 to their respective pin
		configuration. A given digit is displayed by setting PORTD to the respective
		value in the digit mapping array. For example, if 3 were the digit to display
		then PORTD is set to the value stored in the mapping array at index 3. 

		\subsection{Display Digit Interface}
		The two digit number to display is stored as two variables which
		contain the indexes to the mapping array; one index per digit. The value to
		display can then be set by calling \lstinline{setNum()} which handles
		storing a two digit number in the index variables. This function can be
		called at any point in the program to set the display.

	\section{ADC to Temperature Conversion}
		\subsection{Obtaining the ADC Conversion Result (or ADC value)}
		The ADC uses two eight bit registers to store the result from an analog to
		digital conversion - \lstinline{ADCL} which stores the first eight bits of
		the result and \lstinline{ADCH} which stores the final two bits of the
		conversion result. The conversion result is obtained by fetching the
		\lstinline{ADCL} register data first (which is required according to the
		Atmega328p datasheet) and then the \lstinline{ADCH} data is fetched last.
		The final result is stored in a sixteen bit integer which is then passed to
		a temperature conversion function (see the next section). 

		\subsection{Displaying the Temperature}
		A function called \lstinline{setTemp()} handles the conversion of an ADC
		value to a temperature and the display of the resulting conversion. 
\end{document}


